\section{Bài 1}
\textit{Bài làm trong Jupyter Notebook}

\begin{figure}[H]
    \centering
    \includegraphics[width=0.5\textwidth]{img/cau1.png}
    \caption{Mạch lượng tử tạo trạng thái Bell từ trạng thái cơ sở tính toán}
\end{figure}

\vietnameselst
\lstinputlisting[language=Python, caption="Cài đặt mạch lượng tử tạo trạng thái Bell trên Qiskit"]{code/cau1.py}

Kết quả:
% |00>      | (0.707+0j)|00>   (0.707+0j)|11>
% |01>      | (0.707+0j)|01>   (0.707+0j)|10>
% |10>      | (0.707+0j)|00>   (-0.707+0j)|11>
% |11>      | (0.707+0j)|01>   (-0.707+0j)|10>

\begin{table}[H]
    \centering
    \begin{tabular}{|c|c|c|c|}
        \hline
        Input      & Output                                     \\\hline
        $\ket{00}$ & $(0.707+0j)\ket{00} + (0.707+0j)\ket{11}$  \\\hline
        $\ket{01}$ & $(0.707+0j)\ket{01} + (0.707+0j)\ket{10}$  \\\hline
        $\ket{10}$ & $(0.707+0j)\ket{00} + (-0.707+0j)\ket{11}$ \\\hline
        $\ket{11}$ & $(0.707+0j)\ket{01} + (-0.707+0j)\ket{10}$ \\\hline
    \end{tabular}
\end{table}
Mạch lượng tử này chuyển đổi trạng thái cơ sở tính toán sang trạng thái Bell, tức là tạo ra sự vướng víu giữa hai qubit đầu vào.


\section{Bài 2}
\subsection{Câu a}
Một hàm logic là khả nghịch nếu nó là một song ánh.

Quan sát bảng ta thấy không có hàng nào trùng output, do đó hàm này là một song ánh và khả nghịch.

\subsection{Câu b}
Ta tìm hàm Boolean cho $D, E, F$ theo $A, B, C$.

\begin{itemize}
    \item $F$ bằng 1 tại $(0,0,1), (0,1,1), (1,0,1), (1,1,0)$. Ta có biểu thức:
          \begin{align*}
              F & = \overline{A} \overline{B} C + \overline{A} B C + A \overline{B} C + A B \overline{C} \\
                & = \overline{A} C + \overline{B} C + AB\overline{C}                                     \\
                & = AB \oplus C
          \end{align*}
    \item $E$ bằng 1 tại $(0,1,0), (0,1,1), (1,0,0), (1,1,0)$. Ta có biểu thức:
          \begin{align*}
              E & = \overline{A} B  \overline{C} + \overline{A} B C + A \overline{B} \overline{C} + A B \overline{C} \\
                & = \overline{A}B + A\overline{C}
          \end{align*}
    \item $D$ bằng 1 tại $(0,0,0), (0,0,1), (0,1,0), (0,1,1)$. Ta có biểu thức:
          \begin{align*}
              D & = \overline{A} \overline{B} \overline{C} + \overline{A} \overline{B} C + \overline{A} B \overline{C} + \overline{A} B C \\
                & = \overline{A}
          \end{align*}
\end{itemize}

Ta có mạch logic:
\begin{center}
    \begin{circuitikz}[american, scale=0.9, transform shape]
        % Inputs
        \node (A) at (0,0) {$A$};
        \node (B) at (0,-2) {$B$};
        \node (C) at (0,-4) {$C$};

        % NOT gates
        \node[not port, scale=0.6] (notA) at (2,-0.5) {};
        \node[not port, scale=0.6] (notB) at (2,-2.5) {};
        \node[not port, scale=0.6] (notC) at (2,-4.5) {};

        % Connect inputs to NOTs with dots
        \draw (A) -- (1,0) node[circ]{} |- (notA.in);
        \draw (B) -- (1,-2) node[circ]{} |- (notB.in);
        \draw (C) -- (1,-4) node[circ]{} |- (notC.in);

        % NOT outputs with dots
        \draw (notA.out) -- (3.5,-0.5) node[circ]{};
        \draw (notB.out) -- (3.5,-2.5) node[circ]{};
        \draw (notC.out) -- (3.5,-4.5) node[circ]{};

        % D = NOT(A)
        \draw (3.5,-0.5) -- ++(9,0) node[right] {$D$};

        % E = NOT(A)B + A*NOT(C)
        \node[and port, scale=0.6] (andE1) at (6,-6) {};
        \draw (3.5,-0.5) |- (andE1.in 1);
        \draw (1,-2) |- (andE1.in 2);

        \node[and port, scale=0.6] (andE2) at (6,-7.5) {};
        \draw (1,0) |- (andE2.in 1);
        \draw (3.5,-4.5) |- (andE2.in 2);

        \node[or port, scale=0.6] (orE) at (9,-6.75) {};
        \draw (andE1.out) |- (orE.in 1);
        \draw (andE2.out) |- (orE.in 2);
        \draw (orE.out) -- ++(1,0) node[right] {$E$};

        % F = AB XOR C
        \node[and port, scale=0.6] (andAB) at (6,-9) {};
        \draw (1,0) |- (andAB.in 1);
        \draw (1,-2) |- (andAB.in 2);

        \node[xor port, scale=0.6] (xorF) at (9,-9.5) {};
        \draw (andAB.out) |- (xorF.in 1);
        \draw (1,-4) |- (xorF.in 2);
        \draw (xorF.out) -- ++(1,0) node[right] {$F$};
    \end{circuitikz}
\end{center}

\subsection{Câu c}
Dựa trên các hàm Boolean đã rút gọn ở câu b:
\begin{itemize}
    \item $D = \overline{A}$
    \item $E = \overline{A}B + A\overline{C}$
    \item $F = AB \oplus C$
\end{itemize}

Ta thiết kế mạch lượng tử sử dụng 3 qubit đầu vào $\ket{A}, \ket{B}, \ket{C}$ (ký hiệu $q_0, q_1, q_2$) và 3 qubit đầu ra $\ket{D}, \ket{E}, \ket{F}$ (ký hiệu $q_3, q_4, q_5$) được khởi tạo ở trạng thái $\ket{0}$.

Quy trình thực hiện trên mạch (sử dụng các cổng X, CNOT, Toffoli):
\begin{enumerate}
    \item \textbf{Tính F:} Sử dụng cổng Toffoli với điều khiển $q_0, q_1$ lên $q_5$ (tạo $AB$), sau đó dùng CNOT từ $q_2$ lên $q_5$ (tạo $AB \oplus C$).
    \item \textbf{Tính E:}
          \begin{itemize}
              \item Tính $\overline{A}B$: Áp dụng $X(q_0)$, sau đó $Toffoli(q_0, q_1, q_4)$. Sau đó áp dụng $X(q_0)$ để trả lại trạng thái $A$.
              \item Tính $A\overline{C}$: Áp dụng $X(q_2)$, sau đó $Toffoli(q_0, q_2, q_4)$. Sau đó áp dụng $X(q_2)$ để trả lại trạng thái $C$.
          \end{itemize}
    \item \textbf{Tính D:} Áp dụng $X(q_0)$, sau đó $CNOT(q_0, q_3)$. Cuối cùng áp dụng $X(q_0)$ để trả lại trạng thái $A$ ban đầu.
\end{enumerate}

\vietnameselst
\lstinputlisting[language=Python, caption="Cài đặt mạch lượng tử cho hàm f trên Qiskit"]{code/cau2.py}

\begin{figure}[H]
    \centering
    \includegraphics[width=0.8\textwidth]{img/cau2.png}
    \caption{Mạch lượng tử thực hiện hàm $f(A,B,C) = (D,E,F)$}
\end{figure}

\subsection{Câu d}
Trạng thái đầu vào là:
$$ \ket{\psi}_{in} = \frac{1}{\sqrt{3}}(\ket{000} + \ket{100} + \ket{111}) $$
Ta xác định đầu ra cho từng thành phần cơ sở dựa trên hàm $f$:
\begin{itemize}
    \item Với $\ket{000}$ ($A=0, B=0, C=0$):
          \begin{itemize}
              \item $D = \overline{0} = 1$
              \item $E = \overline{0}\cdot 0 + 0\cdot \overline{0} = 0$
              \item $F = 0\cdot 0 \oplus 0 = 0$
              \item Kết quả: $\ket{100}$
          \end{itemize}

    \item Với $\ket{100}$ ($A=1, B=0, C=0$):
          \begin{itemize}
              \item $D = \overline{1} = 0$
              \item $E = \overline{1}\cdot 0 + 1\cdot \overline{0} = 1$
              \item $F = 1\cdot 0 \oplus 0 = 0$
              \item Kết quả: $\ket{010}$
          \end{itemize}

    \item Với $\ket{111}$ ($A=1, B=1, C=1$):
          \begin{itemize}
              \item $D = \overline{1} = 0$
              \item $E = \overline{1}\cdot 1 + 1\cdot \overline{1} = 0$
              \item $F = 1\cdot 1 \oplus 1 = 0$
              \item Kết quả: $\ket{000}$
          \end{itemize}
\end{itemize}

Vậy trạng thái đầu ra của hệ 3 qubit chứa giá trị (D, E, F) là:
$$ \ket{\psi}_{out} = \frac{1}{\sqrt{3}}(\ket{100} + \ket{010} + \ket{000}) $$

Kiểm tra lại bằng mạch lượng tử trên Qiskit đã cho kết quả đúng như trên (xem trong notebook).


\section{Bài 3}
Theo \textbf{Định lý không nhân bản (No-Cloning Theorem)}, ta không thể sao chép một trạng thái lượng tử \textit{bất kỳ chưa biết}. Tuy nhiên, $|\Phi^{+}\rangle$ là một trạng thái \textit{đã biết} (trạng thái Bell). Do đó, việc "sao chép" ở đây được hiểu là chuẩn bị (prepare) một trạng thái giống hệt như vậy trên một cặp qubit khác độc lập.

Trạng thái Bell $\ket{\Phi^{+}}$ được định nghĩa là:
$$ \ket{\Phi^{+}} = \frac{1}{\sqrt{2}}(\ket{00} + \ket{11}) $$

Mạch tạo trạng thái này bao gồm một cổng Hadamard ($H$) tác động lên qubit điều khiển và một cổng CNOT tác động lên qubit mục tiêu. Để tạo ra hai bản sao, ta thực hiện quy trình này trên hai cặp qubit riêng biệt (ví dụ: cặp $q_0, q_1$ và cặp $q_2, q_3$).

Trạng thái kỳ vọng của hệ 4 qubit là tích tensor của hai trạng thái Bell:
\begin{align*}
    \ket{\Psi} & = \ket{\Phi^{+}}_{23} \otimes \ket{\Phi^{+}}_{01}                                                                       \\
               & = \left[ \frac{1}{\sqrt{2}}(\ket{00} + \ket{11}) \right] \otimes \left[ \frac{1}{\sqrt{2}}(\ket{00} + \ket{11}) \right] \\
               & = \frac{1}{2} \left( \ket{0000} + \ket{0011} + \ket{1100} + \ket{1111} \right)
\end{align*}

Mạch lượng tử:
\begin{figure}[H]
    \centering
    \includegraphics[width=0.3\textwidth]{img/cau3.png}
    \caption{Mạch lượng tử tạo hai bản sao của trạng thái Bell $\ket{\Phi^{+}}$}
\end{figure}

Kiểm tra kết quả trên Qiskit cho thấy trạng thái đầu ra đúng như kỳ vọng:
$$ \ket{\Psi} = \frac{1}{2}(\ket{0000} + \ket{0011} + \ket{1100} + \ket{1111}) $$
\begin{itemize}
    \item Cặp bit thấp ($q_1 q_0$) nhận giá trị $00$ hoặc $11$, tương ứng với trạng thái $|\Phi^{+}\rangle$.
    \item Cặp bit cao ($q_3 q_2$) nhận giá trị $00$ hoặc $11$, cũng tương ứng với trạng thái $|\Phi^{+}\rangle$.
    \item Sự xuất hiện của các tổ hợp chéo (ví dụ $|00\rangle_{23}|11\rangle_{01}$) cho thấy hai cặp này độc lập với nhau.
\end{itemize}
Như vậy, ta đã tạo thành công bản sao của trạng thái $|\Phi^{+}\rangle$.

\section{Bài 4}
% Viết lời giải ở đây.

\section{Bài 5}
% Viết lời giải ở đây.
