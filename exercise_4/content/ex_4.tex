\section{Câu 4}
\textbf{Đề bài:} Cài đặt hoàn chỉnh thuật toán Grover và thực hiện các tính toán cho trường hợp $n=4$ với lời giải là $0^{n}$.

\textit{Bài làm:}\\
\textbf{1. Phân tích tham số}
\begin{itemize}
    \item Số qubit: $n = 4$. Không gian tìm kiếm $N = 2^4 = 16$.
    \item Trạng thái cần tìm: $\ket{w} = \ket{0000}$.
    \item Số lần lặp tối ưu (Grover iterations): $k \approx \frac{\pi}{4}\sqrt{N} \approx 3.14 \Rightarrow$ chọn $\mathbf{k=3}$.
\end{itemize}

\textbf{2. Thiết kế mạch}
Mạch bao gồm khởi tạo trạng thái chồng chập $\ket{s} = H^{\otimes 4}\ket{0}^{\otimes 4}$, theo sau là 3 lần lặp của toán tử Grover $G = U_s U_w$.

\textbf{a) Oracle ($U_w$):}
Đảo dấu pha của trạng thái $\ket{0000}$. Ta dùng cổng Multi-controlled Z (MCZ). Vì MCZ chuẩn tác động lên $\ket{1111}$, ta cần bọc nó bởi các cổng $X$ (NOT).
$$ U_w = X^{\otimes 4} \cdot \text{MCZ} \cdot X^{\otimes 4} $$

\textbf{b) Diffuser ($U_s$):}
Phản xạ qua trạng thái trung bình $\ket{s}$. Công thức là $H^{\otimes 4}(2\ket{0}\bra{0} - I)H^{\otimes 4}$. Phần lõi $(2\ket{0}\bra{0} - I)$ chính là mạch phản xạ quanh $\ket{0000}$ tương tự như Oracle.

\vspace{0.5cm}
\textbf{Sơ đồ mạch (Minh họa 1 vòng lặp):}
\begin{center}
    \begin{quantikz}
        \lstick{$\ket{0}$} & \gate{H} & \gate{X} & \ctrl{3} & \gate{X} & \gate{H} & \gate{X} & \ctrl{3} & \gate{X} & \gate{H} & \qw \\
        \lstick{$\ket{0}$} & \gate{H} & \gate{X} & \ctrl{2} & \gate{X} & \gate{H} & \gate{X} & \ctrl{2} & \gate{X} & \gate{H} & \qw \\
        \lstick{$\ket{0}$} & \gate{H} & \gate{X} & \ctrl{1} & \gate{X} & \gate{H} & \gate{X} & \ctrl{1} & \gate{X} & \gate{H} & \qw \\
        \lstick{$\ket{0}$} & \gate{H} & \gate{X} & \gate{Z} & \gate{X} & \gate{H} & \gate{X} & \gate{Z} & \gate{X} & \gate{H} & \qw
    \end{quantikz}
\end{center}

Cài đặt trong Qiskit như sau:
\begin{lstlisting}[language=Python]
def grover():
    n = 4
    qc = QuantumCircuit(n)

    # 1. Khởi tạo |s>
    qc.h(range(n))
    
    # Số lần lặp tối ưu cho N=16 là 3
    for _ in range(3):
        # --- ORACLE: Mark |0000> ---
        # (X -> MCZ -> X)
        qc.x(range(n))
        qc.mcp(np.pi, list(range(n-1)), n-1)
        qc.x(range(n))
        
        # --- DIFFUSER: Reflect |s> ---
        # (H -> X -> MCZ -> X -> H)
        qc.h(range(n))
        qc.x(range(n))
        qc.mcp(np.pi, list(range(n-1)), n-1)
        qc.x(range(n))
        qc.h(range(n))

    return qc
\end{lstlisting}

Chạy thuật toán và kiểm tra kết quả:
\begin{lstlisting}[language=Python]
# 1. Tạo mạch
qc = grover()

# 2. Tính toán Statevector
final_state = Statevector(qc)

# 3. Lấy mẫu kết quả
counts = final_state.sample_counts(shots=1000)

# 4. In kết quả
print("Kết quả")
sorted_counts = sorted(counts.items(), key=lambda x: x[1], reverse=True)

for state, count in sorted_counts:
    prob = (count / 1000) * 100
    print(f"Trạng thái |{state}>: {count} lần ({prob:.1f}%)")

# Lấy biên độ chính xác của |0000>
amp_0000 = final_state.data[0]
prob_theoretical = np.abs(amp_0000)**2 * 100
print("-" * 30)
print(f"Xác suất lý thuyết của |0000>: {prob_theoretical:.2f}%")
\end{lstlisting}

Kết quả thuật toán được trình bày trong Bảng \ref{tab:grover_results}.
\begin{table}[H]
    \centering
    \caption{Kết quả thực nghiệm thuật toán Grover ($n=4$, 1000 shots)}
    \vspace{0.2cm}
    \begin{tabular}{|c|c|c|}
        \hline
        \textbf{Trạng thái}                                         & \textbf{Số lần đo (Counts)} & \textbf{Tần suất (\%)} \\
        \hline
        $|0000\rangle$                                              & \textbf{959}                & \textbf{95.9\%}        \\
        \hline
        $|0111\rangle$                                              & 6                           & 0.6\%                  \\
        $|1011\rangle$                                              & 5                           & 0.5\%                  \\
        $|0010\rangle$                                              & 4                           & 0.4\%                  \\
        $|1001\rangle$                                              & 4                           & 0.4\%                  \\
        $|1010\rangle$                                              & 4                           & 0.4\%                  \\
        $|0001\rangle$                                              & 3                           & 0.3\%                  \\
        $|1111\rangle$                                              & 3                           & 0.3\%                  \\
        $|0011\rangle$                                              & 2                           & 0.2\%                  \\
        $|0100\rangle$                                              & 2                           & 0.2\%                  \\
        $|0110\rangle$                                              & 2                           & 0.2\%                  \\
        $|1000\rangle$                                              & 2                           & 0.2\%                  \\
        $|1100\rangle$                                              & 2                           & 0.2\%                  \\
        $|0101\rangle$                                              & 1                           & 0.1\%                  \\
        $|1110\rangle$                                              & 1                           & 0.1\%                  \\
        \hline
        \multicolumn{3}{|c|}{\textbf{So sánh lý thuyết}}                                                                   \\
        \hline
        \multicolumn{2}{|l|}{Xác suất lý thuyết của $|0000\rangle$} & 96.13\%                                              \\
        \hline
    \end{tabular}
    \label{tab:grover_results}
\end{table}

