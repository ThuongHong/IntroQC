\section{Câu 5}
\textbf{Đề bài:} Thiết kế mạch QFT 3 qubit. Tính toán từng bước để thấy mạch biến các vector cơ sở tính toán $\ket{j}$ thành các vector $\ket{\phi_j}$ tương ứng.\\
\textit{Bài làm:}

\textbf{1. Thiết kế mạch}
Mạch QFT cho 3 qubit ($q_1, q_2, q_3$ tương ứng với các bit của đầu vào $|j\rangle = |j_1 j_2 j_3\rangle$) sử dụng các cổng Hadamard ($H$) và Controlled-Phase $R_k$ (với pha $2\pi/2^k$). Sơ đồ mạch như sau:

\begin{center}
    \begin{quantikz}
        \lstick{$\ket{j_1}$} & \gate{H} & \gate{R_2} & \gate{R_3} & \qw      & \qw        & \qw      & \swap{2} & \rstick{$\ket{\phi_1}$} \qw \\
        \lstick{$\ket{j_2}$} & \qw      & \ctrl{-1}  & \qw        & \gate{H} & \gate{R_2} & \qw      & \qw      & \rstick{$\ket{\phi_2}$} \qw \\
        \lstick{$\ket{j_3}$} & \qw      & \qw        & \ctrl{-2}  & \qw      & \ctrl{-1}  & \gate{H} & \swap{-2}& \rstick{$\ket{\phi_3}$} \qw
    \end{quantikz}
\end{center}

\textbf{2. Tính toán từng bước}
Xét trạng thái đầu vào $\ket{j} = \ket{j_1 j_2 j_3}$. Ta biểu diễn dưới dạng thập phân $j = j_1 2^2 + j_2 2^1 + j_3 2^0 = 4j_1 + 2j_2 + j_3$.

\textbf{Bước 1: Biến đổi trên qubit 1 ($\ket{j_1}$)}
\begin{itemize}
    \item Áp dụng $H$: $\frac{1}{\sqrt{2}}(\ket{0} + e^{2\pi i (0.j_1)}\ket{1})$.
    \item Áp dụng $R_2$ (điều khiển bởi $j_2$): Thêm pha $2\pi i \frac{j_2}{2^2} = 2\pi i (0.0j_2)$.
    \item Áp dụng $R_3$ (điều khiển bởi $j_3$): Thêm pha $2\pi i \frac{j_3}{2^3} = 2\pi i (0.00j_3)$.
\end{itemize}
Trạng thái qubit 1 lúc này: $\frac{1}{\sqrt{2}}(\ket{0} + e^{2\pi i (0.j_1 j_2 j_3)}\ket{1})$.

\textbf{Bước 2: Biến đổi trên qubit 2 ($\ket{j_2}$)}
\begin{itemize}
    \item Áp dụng $H$: $\frac{1}{\sqrt{2}}(\ket{0} + e^{2\pi i (0.j_2)}\ket{1})$.
    \item Áp dụng $R_2$ (điều khiển bởi $j_3$): Thêm pha $2\pi i (0.0j_3)$.
\end{itemize}
Trạng thái qubit 2 lúc này: $\frac{1}{\sqrt{2}}(\ket{0} + e^{2\pi i (0.j_2 j_3)}\ket{1})$.

\textbf{Bước 3: Biến đổi trên qubit 3 ($\ket{j_3}$)}
\begin{itemize}
    \item Chỉ áp dụng $H$: $\frac{1}{\sqrt{2}}(\ket{0} + e^{2\pi i (0.j_3)}\ket{1})$.
\end{itemize}

\textbf{Bước 4: Cổng SWAP}
Mạch QFT yêu cầu đảo ngược thứ tự qubit để phù hợp với định nghĩa toán học. Ta đổi chỗ qubit 1 và qubit 3.
Kết quả cuối cùng là tích tensor:
$$ \ket{\psi} = \frac{1}{\sqrt{8}}
    \left(\ket{0} + e^{2\pi i 0.j_3}\ket{1}\right) \otimes
    \left(\ket{0} + e^{2\pi i 0.j_2 j_3}\ket{1}\right) \otimes
    \left(\ket{0} + e^{2\pi i 0.j_1 j_2 j_3}\ket{1}\right) $$
Đây chính xác là biểu diễn của QFT dưới dạng tích tensor (product state).

Cài đặt bằng Qiskit:
\begin{lstlisting}[language=Python]
def qft_3_qubit():    
    qc = QuantumCircuit(3)
    
    # --- Thiết kế thủ công (Manual Design) ---
    # 1. H trên q2 (tương ứng j1 trong bài làm)
    qc.h(2)
    # Controlled-Phase từ q1 lên q2 (góc pi/2)
    qc.cp(np.pi/2, 1, 2)
    # Controlled-Phase từ q0 lên q2 (góc pi/4)
    qc.cp(np.pi/4, 0, 2)
    
    # 2. H trên q1 (tương ứng j2)
    qc.h(1)
    # Controlled-Phase từ q0 lên q1 (góc pi/2)
    qc.cp(np.pi/2, 0, 1)
    
    # 3. H trên q0 (tương ứng j3)
    qc.h(0)
    
    # 4. SWAP đầu cuối
    qc.swap(0, 2)
    
    return qc
\end{lstlisting}

Chạy và kiểm tra:
\begin{lstlisting}[language=Python]
# 1. Tạo mạch thủ công
manual_qc = qft_3_qubit()

# 2. Tạo mạch chuẩn thư viện để so sánh
library_qc = QFT(num_qubits=3, do_swaps=True).decompose()

# 3. Tính toán statevector (Ma trận toán tử)
# Ta kiểm tra xem 2 mạch có tạo ra cùng một toán tử Unitary không
sv_manual = Statevector(manual_qc)
sv_lib = Statevector(library_qc) # Input mặc định là |000>

# Tính độ tương đồng (Fidelity)
fidelity = sv_manual.inner(sv_lib)
print(f"Độ trùng khớp (Fidelity): {abs(fidelity):.5f}")

# In thử kết quả với input |000>
print("Statevector output:")
print(np.round(sv_manual.data, 3))
\end{lstlisting}

Kết quả thu được:
\begin{lstlisting}[language=Python]
Độ trùng khớp (Fidelity): 1.00000
Statevector output:
[0.354+0.j 0.354+0.j 0.354+0.j 0.354+0.j 0.354+0.j 0.354+0.j 0.354+0.j
 0.354+0.j]
\end{lstlisting}
Như vậy, mạch QFT thủ công đã được thiết kế đúng và cho kết quả trùng khớp hoàn toàn với mạch QFT chuẩn từ thư viện Qiskit.