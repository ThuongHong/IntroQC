\section{Bài 1}
\textbf{Đề bài:} Khảo sát phép đo theo các cơ sở $B_{Z}=\{|0\rangle,|1\rangle\}$, $B_{X}=\{|+\rangle,|-\rangle\}$, $B_{Y}=\{|i\rangle,|-i\rangle\}$ của các trạng thái lượng tử sau.

\subsection{(a)}
\textbf{Đề bài:} $|\psi_{1}\rangle = \frac{\sqrt{3}}{2}|0\rangle + \frac{1}{2}|1\rangle$.\\
\textbf{Bài làm:}\\
\textbf{Cơ sở $B_{Z}$:} $[|\psi_1\rangle]_{B_{Z}} = \begin{pmatrix}
        \frac{\sqrt{3}}{2},
        \frac{1}{2}
    \end{pmatrix}$.\\
Khi đo $|\psi_1\rangle$ theo cơ sở $B_{Z}$ thì sẽ được $|0\rangle$ với xác suất là $\left|\frac{\sqrt{3}}{2}\right|^2 = \frac{3}{4}$, được $|1\rangle$ với xác suất là $\left|\frac{1}{2}\right|^2 = \frac{1}{4}$.

\textbf{Cơ sở $B_{X}$:} Trạng thái $|\psi_1\rangle$ biểu diễn theo cơ sở $B_{X}$ như sau:
\begin{align*}
    |\psi_1\rangle & = \langle+|\psi_1\rangle |+\rangle + \langle-|\psi_1\rangle |-\rangle                                       \\
                   & = \left(\frac{\sqrt{3}}{2} \cdot \frac{1}{\sqrt{2}} + \frac{1}{2} \cdot \frac{1}{\sqrt{2}}\right) |+\rangle
    + \left(\frac{\sqrt{3}}{2} \cdot \frac{1}{\sqrt{2}} - \frac{1}{2} \cdot \frac{1}{\sqrt{2}}\right) |-\rangle                  \\
                   & = \frac{\sqrt{3} + 1}{2\sqrt{2}} |+\rangle + \frac{\sqrt{3} - 1}{2\sqrt{2}} |-\rangle
\end{align*}
Khi đo $|\psi_1\rangle$ theo cơ sở $B_{X}$ thì sẽ được $|+\rangle$ với xác suất là $\left|\frac{\sqrt{3} + 1}{2\sqrt{2}}\right|^2 = \frac{2 + \sqrt{3}}{4}$, được $|-\rangle$ với xác suất là $\left|\frac{\sqrt{3} - 1}{2\sqrt{2}}\right|^2 = \frac{2 - \sqrt{3}}{4}$.

\textbf{Cơ sở $B_{Y}$:} Trạng thái $|\psi_1\rangle$ biểu diễn theo cơ sở $B_{Y}$ như sau:
\begin{align*}
    |\psi_1\rangle & = \langle i|\psi_1\rangle |i\rangle + \langle -i|\psi_1\rangle |-i\rangle                                    \\
                   & = \left(\frac{\sqrt{3}}{2} \cdot \frac{1}{\sqrt{2}} + \frac{1}{2} \cdot \frac{-i}{\sqrt{2}}\right) |i\rangle
    + \left(\frac{\sqrt{3}}{2} \cdot \frac{1}{\sqrt{2}} + \frac{1}{2} \cdot \frac{i}{\sqrt{2}}\right) |-i\rangle                  \\
                   & = \frac{\sqrt{3} - i}{2\sqrt{2}} |i\rangle + \frac{\sqrt{3} + i}{2\sqrt{2}} |-i\rangle
\end{align*}
Khi đo $|\psi_1\rangle$ theo cơ sở $B_{Y}$ thì sẽ được $|i\rangle$ với xác suất là $\left|\frac{\sqrt{3} - i}{2\sqrt{2}}\right|^2 = \frac{1}{2}$, được $|-i\rangle$ với xác suất là $\left|\frac{\sqrt{3} + i}{2\sqrt{2}}\right|^2 = \frac{1}{2}$.


\subsection{(b)}
\textbf{Đề bài:} $|\psi_{2}\rangle=\frac{1}{\sqrt{2}}(|0\rangle+e^{i\frac{\pi}{6}}|1\rangle)$.\\
\textbf{Bài làm:}\\
\textbf{Cơ sở $B_{Z}$:} $[|\psi_2\rangle]_{B_{Z}} = \begin{pmatrix}
        \frac{1}{\sqrt{2}},
        \frac{1}{\sqrt{2}} e^{i\frac{\pi}{6}}
    \end{pmatrix}$.\\
Khi đo $|\psi_2\rangle$ theo cơ sở $B_{Z}$ thì sẽ được $|0\rangle$ với xác suất là $\left|\frac{1}{\sqrt{2}}\right|^2 = \frac{1}{2}$, được $|1\rangle$ với xác suất là $\left|\frac{1}{\sqrt{2}} e^{i\frac{\pi}{6}}\right|^2 = \frac{1}{2}$.

\textbf{Cơ sở $B_{X}$:} Trạng thái $|\psi_2\rangle$ biểu diễn theo cơ sở $B_{X}$ như sau:
\begin{align*}
    |\psi_2\rangle & = \langle+|\psi_2\rangle |+\rangle + \langle-|\psi_2\rangle |-\rangle                                                                                                                                                                                                      \\
                   & = \left(\frac{1}{\sqrt{2}} \cdot \frac{1}{\sqrt{2}} + \frac{1}{\sqrt{2}} e^{i\frac{\pi}{6}} \cdot \frac{1}{\sqrt{2}}\right) |+\rangle + \left(\frac{1}{\sqrt{2}} \cdot \frac{1}{\sqrt{2}} - \frac{1}{\sqrt{2}}e^{i\frac{\pi}{6}} \cdot \frac{1}{\sqrt{2}}\right) |-\rangle \\
                   & = \left(\frac{1}{2} + \frac{1}{2} e^{i\frac{\pi}{6}}\right)|+\rangle + \left(\frac{1}{2} - \frac{1}{2} e^{i\frac{\pi}{6}}\right)|-\rangle
\end{align*}
Khi đo $|\psi_2\rangle$ theo cơ sở $B_{X}$ thì sẽ được $|+\rangle$ với xác suất là $\left|\frac{1}{2} + \frac{1}{2} e^{i\frac{\pi}{6}}\right|^2 = \frac{2 + \sqrt{3}}{4}$, được $|-\rangle$ với xác suất là $\left|\frac{1}{2} - \frac{1}{2} e^{i\frac{\pi}{6}}\right|^2 = \frac{2 - \sqrt{3}}{4}$.

\textbf{Cơ sở $B_{Y}$:} Trạng thái $|\psi_2\rangle$ biểu diễn theo cơ sở $B_{Y}$ như sau:
\begin{align*}
    |\psi_2\rangle & = \langle i|\psi_2\rangle |i\rangle + \langle -i|\psi_2\rangle |-i\rangle                                                                                                                                                                                                     \\
                   & = \left(\frac{1}{\sqrt{2}} \cdot \frac{1}{\sqrt{2}} + \frac{1}{\sqrt{2}} e^{i\frac{\pi}{6}} \cdot \frac{-i}{\sqrt{2}}\right) |i\rangle + \left(\frac{1}{\sqrt{2}} \cdot \frac{1}{\sqrt{2}} + \frac{1}{\sqrt{2}} e^{i\frac{\pi}{6}} \cdot \frac{i}{\sqrt{2}}\right) |-i\rangle \\
                   & = \left(\frac{1}{2} - \frac{i}{2} e^{i\frac{\pi}{6}}\right)|i\rangle + \left(\frac{1}{2} + \frac{i}{2} e^{i\frac{\pi}{6}}\right)|-i\rangle
\end{align*}
Khi đo $|\psi_2\rangle$ theo cơ sở $B_{Y}$ thì sẽ được $|i\rangle$ với xác suất là $\left|\frac{1}{2} - \frac{i}{2} e^{i\frac{\pi}{6}}\right|^2 = \frac{3}{4}$, được $|-i\rangle$ với xác suất là $\left|\frac{1}{2} + \frac{i}{2} e^{i\frac{\pi}{6}}\right|^2 = \frac{1}{4}$.

\subsection{(c)}
\textbf{Đề bài:} $|\psi_{3}\rangle=\frac{2}{3}|0\rangle+\frac{1-2i}{3}|1\rangle$.\\
\textbf{Bài làm:}\\
\textbf{Cơ sở $B_{Z}$:} $[|\psi_3\rangle]_{B_{Z}} = \begin{pmatrix}
        \frac{2}{3},
        \frac{1-2i}{3}
    \end{pmatrix}$.\\
Khi đo $|\psi_3\rangle$ theo cơ sở $B_{Z}$ thì sẽ được $|0\rangle$ với xác suất là $\left|\frac{2}{3}\right|^2 = \frac{4}{9}$, được $|1\rangle$ với xác suất là $\left|\frac{1-2i}{3}\right|^2 = \frac{5}{9}$.

\textbf{Cơ sở $B_{X}$:} Trạng thái $|\psi_3\rangle$ biểu diễn theo cơ sở $B_{X}$ như sau:
\begin{align*}
    |\psi_3\rangle & = \langle+|\psi_3\rangle |+\rangle + \langle-|\psi_3\rangle |-\rangle                                                                                                                                           \\
                   & = \left(\frac{2}{3} \cdot \frac{1}{\sqrt{2}} + \frac{1-2i}{3} \cdot \frac{1}{\sqrt{2}}\right) |+\rangle + \left(\frac{2}{3} \cdot \frac{1}{\sqrt{2}} - \frac{1-2i}{3} \cdot \frac{1}{\sqrt{2}}\right) |-\rangle \\
                   & = \left(\frac{3 - 2i}{3\sqrt{2}}\right)|+\rangle + \left(\frac{1 + 2i}{3\sqrt{2}}\right)|-\rangle
\end{align*}
Khi đo $|\psi_3\rangle$ theo cơ sở $B_{X}$ thì sẽ được $|+\rangle$ với xác suất là $\left|\frac{3 - 2i}{3\sqrt{2}}\right|^2 = \frac{13}{18}$, được $|-\rangle$ với xác suất là $\left|\frac{1 + 2i}{3\sqrt{2}}\right|^2 = \frac{5}{18}$.

\textbf{Cơ sở $B_{Y}$:} Trạng thái $|\psi_3\rangle$ biểu diễn theo cơ sở $B_{Y}$ như sau:
\begin{align*}
    |\psi_3\rangle & = \langle i|\psi_3\rangle |i\rangle + \langle -i|\psi_3\rangle |-i\rangle                                                                                                                                         \\
                   & = \left(\frac{2}{3} \cdot \frac{1}{\sqrt{2}} + \frac{1-2i}{3} \cdot \frac{-i}{\sqrt{2}}\right) |i\rangle + \left(\frac{2}{3} \cdot \frac{1}{\sqrt{2}} + \frac{1-2i}{3} \cdot \frac{i}{\sqrt{2}}\right) |-i\rangle \\
                   & = \left(-\frac{\sqrt{2}}{6}i\right)|i\rangle + \left(\frac{2\sqrt{2}}{3}+\frac{\sqrt{2}}{6}i\right)|-i\rangle
\end{align*}
Khi đo $|\psi_3\rangle$ theo cơ sở $B_{Y}$ thì sẽ được $|i\rangle$ với xác suất là $\left|-\frac{\sqrt{2}}{6}i\right|^2 = \frac{1}{18}$, được $|-i\rangle$ với xác suất là $\left|\frac{2\sqrt{2}}{3}+\frac{\sqrt{2}}{6}i\right|^2 = \frac{17}{18}$.

\section{Bài 2}
\textbf{Đề bài:} Viết dạng Bloch và mô tả trên mặt cầu Bloch các trạng thái lượng tử ở Câu 1.\\
\textbf{Bài làm:}\\
\textbf{Trạng thái $\ket{\psi_1}$}:
\begin{align*}
    \ket{\psi_1} & = \frac{\sqrt{3}}{2}\ket{0} + \frac{1}{2}\ket{1} = \cos\frac{\pi}{6}\ket{0} + e^{i0}\sin\frac{\pi}{6}\ket{1} \\
                 & \implies \theta = \frac{\pi}{3}, \phi = 0
\end{align*}
Hình vẽ:
\begin{figure}[H]
    \centering
    \includegraphics[width=0.6\textwidth]{img/psi_1.png}
    \caption{Trạng thái $\ket{\psi_1}$ trên mặt cầu Bloch}
\end{figure}

\textbf{Trạng thái $\ket{\psi_2}$}:
\begin{align*}
    \ket{\psi_2} & = \frac{1}{\sqrt{2}}\ket{0} + \frac{1}{\sqrt{2}}e^{i\frac{\pi}{6}}\ket{1} = \cos\frac{\pi}{4}\ket{0} + e^{i\frac{\pi}{6}}\sin\frac{\pi}{4}\ket{1} \\
                 & \implies \theta = \frac{\pi}{2}, \phi = \frac{\pi}{6}
\end{align*}
Hình vẽ:
\begin{figure}[H]
    \centering
    \includegraphics[width=0.6\textwidth]{img/psi_2.png}
    \caption{Trạng thái $\ket{\psi_2}$ trên mặt cầu Bloch}
\end{figure}

\textbf{Trạng thái $\ket{\psi_3}$}:
\begin{align*}
    \ket{\psi_3} & = \frac{2}{3}\ket{0} + \frac{1-2i}{3}\ket{1} = \cos\frac{\theta}{2}\ket{0} + e^{i\phi}\sin\frac{\theta}{2}\ket{1}                                                                                                                    \\
                 & \implies \begin{cases}
                                \cos\frac{\theta}{2} = \frac{2}{3} \\
                                e^{i\phi}\sin\frac{\theta}{2} = \frac{1-2i}{3}
                            \end{cases}                                                                                                                                                                               \\
                 & \implies \begin{cases}
                                \cos\frac{\theta}{2} = \frac{2}{3} \\
                                e^{i\phi} = \frac{\frac{1-2i}{3}}{\sqrt{1 - \cos^2{\frac{\theta}{2}}}} = \frac{\frac{1-2i}{3}}{\sqrt{1 - \left(\frac{2}{3}\right)^2}} = \frac{\frac{1-2i}{3}}{\sqrt{\frac{5}{9}}} = \frac{1-2i}{\sqrt{5}}
                            \end{cases} \\
                 & \implies \begin{cases}
                                \theta = 2\arccos\frac{2}{3} \approx 0.841 \\
                                \phi = \arg{\frac{1-2i}{\sqrt{5}}} = \arctan{\left(-2\right)} \approx -1.107
                            \end{cases}
\end{align*}
Hình vẽ:
\begin{figure}[H]
    \centering
    \includegraphics[width=0.6\textwidth]{img/psi_3.png}
    \caption{Trạng thái $\ket{\psi_3}$ trên mặt cầu Bloch}
\end{figure}

\section{Bài 3}
\textbf{Đề bài:} Cho $U$ là một toán tử tuyến tính trên $\mathbb{C}^{2}$, biết
$U|0\rangle=\frac{\sqrt{2}-i}{2}|0\rangle-\frac{1}{2}|1\rangle$,
$U|1\rangle=\frac{1}{2}|0\rangle+\frac{\sqrt{2}+i}{2}|1\rangle$.
Ma trận biểu diễn của $U$ theo cơ sở tính toán là:
\begin{align*}
    U =\begin{bmatrix}
           U\ket{0} & U\ket{1}
       \end{bmatrix}
    =\frac{1}{2}\begin{bmatrix}
                    \sqrt{2}-i & 1          \\
                    -1         & \sqrt{2}+i
                \end{bmatrix}
\end{align*}

\subsection{(a)}
\textbf{Đề bài:} Chứng minh $U$ là một cổng lượng tử.\\
\textbf{Bài làm:}
Vì $U$ là toán tử tuyến tính trên $\mathbb{C}^{2}$ nên đã thoả điều kiện:
\begin{align*}
    U(\alpha\ket{\phi} + \beta\ket{\psi}) = \alpha U\ket{\phi} + \beta U\ket{\psi}, \quad \forall \alpha,\beta \in \mathbb{C}, \forall \ket{\phi},\ket{\psi} \in \mathbb{C}^{2}
\end{align*}
Ta cần chứng minh:
\begin{align*}
     & ||U\ket{\phi}|| = ||\ket{\phi}||, \quad \forall \ket{\phi} \in \mathbb{C}^{2} \\
     & \implies U \text{bảo toàn chuẩn, hay } U \text{ là ma trận unita}             \\
     & \implies U^{\dagger}U = I
\end{align*}
Ta có:
\begin{align*}
    U^{\dagger}           & = \frac{1}{2}\begin{bmatrix}
                                             \sqrt{2}+i & -1         \\
                                             1          & \sqrt{2}-i
                                         \end{bmatrix}               \\
    \implies U^{\dagger}U & = \frac{1}{4}\begin{bmatrix}
                                             \sqrt{2}+i & -1         \\
                                             1          & \sqrt{2}-i
                                         \end{bmatrix} \begin{bmatrix}
                                                           \sqrt{2}-i & 1          \\
                                                           -1         & \sqrt{2}+i
                                                       \end{bmatrix}
    = \begin{bmatrix}
          1 & 0 \\
          0 & 1
      \end{bmatrix}
    = I
\end{align*}
Vậy $U$ là ma trận unita, tức là $U$ là một cổng lượng tử.

\subsection{(b)}
\textbf{Đề bài:} Cho biết kết quả biến đổi $U$ trên các trạng thái $|+\rangle,|-\rangle,|i\rangle,|-i\rangle.$\\
\textbf{Bài làm:}
Ta có:
\begin{align*}
    U\ket{+}  & = \frac{1}{2}\begin{bmatrix}
                                 \sqrt{2} - i & 1            \\
                                 -1           & \sqrt{2} + i
                             \end{bmatrix}
    \frac{1}{\sqrt{2}}\begin{bmatrix}
                          1 \\
                          1
                      \end{bmatrix}
    = \frac{1}{2\sqrt{2}}\begin{bmatrix}
                             \sqrt{2} - i + 1 \\
                             \sqrt{2} + i -1
                         \end{bmatrix}                \\
    U\ket{-}  & = \frac{1}{2}\begin{bmatrix}
                                 \sqrt{2} - i & 1            \\
                                 -1           & \sqrt{2} + i
                             \end{bmatrix}
    \frac{1}{\sqrt{2}}\begin{bmatrix}
                          1 \\
                          -1
                      \end{bmatrix}
    = \frac{1}{2\sqrt{2}}\begin{bmatrix}
                             \sqrt{2} - i - 1 \\
                             -\sqrt{2} - i -1
                         \end{bmatrix}                \\
    U\ket{i}  & = \frac{1}{2}\begin{bmatrix}
                                 \sqrt{2} - i & 1            \\
                                 -1           & \sqrt{2} + i
                             \end{bmatrix}
    \frac{1}{\sqrt{2}}\begin{bmatrix}
                          1 \\
                          i
                      \end{bmatrix}
    = \frac{1}{2\sqrt{2}}\begin{bmatrix}
                             \sqrt{2} \\
                             -2 + i\sqrt{2}
                         \end{bmatrix}                  \\
    U\ket{-i} & = \frac{1}{2}\begin{bmatrix}
                                 \sqrt{2} - i & 1            \\
                                 -1           & \sqrt{2} + i
                             \end{bmatrix}
    \frac{1}{\sqrt{2}}\begin{bmatrix}
                          1 \\
                          -i
                      \end{bmatrix}
    = \frac{1}{2\sqrt{2}}\begin{bmatrix}
                             \sqrt{2} - 2i \\
                             -i\sqrt{2}
                         \end{bmatrix}
\end{align*}

\subsection{(c)}
\textbf{Đề bài:} Cho biết kết quả biến đổi $U$ trên các trạng thái của Câu 1.\\
\textbf{Bài làm:}
Ta có:
\begin{align*}
    U\ket{\psi_1} & = \frac{1}{2}\begin{bmatrix}
                                     \sqrt{2} - i & 1            \\
                                     -1           & \sqrt{2} + i
                                 \end{bmatrix}
    \begin{bmatrix}
        \frac{\sqrt{3}}{2} \\
        \frac{1}{2}
    \end{bmatrix}
    = \frac{1}{4}\begin{bmatrix}
                     1 + \sqrt{6} - i\sqrt{3} \\
                     -\sqrt{3} + \sqrt{2} + i
                 \end{bmatrix}                               \\
    U\ket{\psi_2} & = \frac{1}{2}\begin{bmatrix}
                                     \sqrt{2} - i & 1            \\
                                     -1           & \sqrt{2} + i
                                 \end{bmatrix}
    \begin{bmatrix}
        \frac{1}{\sqrt{2}} \\
        \frac{1}{\sqrt{2}} e^{i\frac{\pi}{6}}
    \end{bmatrix}
    = \frac{1}{2\sqrt{2}}\begin{bmatrix}
                             \sqrt{2} - i + e^{i\frac{\pi}{6}} \\
                             -1 + (\sqrt{2} + i)e^{i\frac{\pi}{6}}
                         \end{bmatrix} \\
    U\ket{\psi_3} & = \frac{1}{2}\begin{bmatrix}
                                     \sqrt{2} - i & 1            \\
                                     -1           & \sqrt{2} + i
                                 \end{bmatrix}
    \begin{bmatrix}
        \frac{2}{3} \\
        \frac{1-2i}{3}
    \end{bmatrix}
    = \frac{1}{6}\begin{bmatrix}
                     1 + 2\sqrt{2} -4i \\
                     \sqrt{2} + (1 - 2\sqrt{2})i
                 \end{bmatrix}
\end{align*}

\subsection{(d)}
\textbf{Đề bài:} $U$ tương ứng với phép quay quanh trục nào với góc bao nhiêu trên mặt cầu Bloch?\\
\textbf{Bài làm:}\\
Vì $U$ unita nên luôn tìm được 2 vector riêng $\ket{u_1},\ket{u_2}$ lập thành một
cơ sở trực chuẩn với các trị riêng tương ứng là $e^{i\theta_1},e^{i\theta_2}$, khi đó:
\begin{align*}
    U\ket{u_1} & = e^{i\theta_1}\ket{u_1} \equiv \ket{u_1} ,\quad U\ket{u_2} = e^{i\theta_2}\ket{u_2} \equiv \ket{u_2}
\end{align*}
Ta cần tìm 2 trị riêng và vector riêng của $U$.\\
Ta có ma trận đặc trưng:
\begin{align*}
    U - \lambda I       & = \frac{1}{2}\begin{bmatrix}
                                           \sqrt{2} - i - 2\lambda & 1                       \\
                                           -1                      & \sqrt{2} + i - 2\lambda
                                       \end{bmatrix}                                                   \\
    \det(U - \lambda I) & = \frac{1}{4}[(\sqrt{2} - i - 2\lambda)(\sqrt{2} + i - 2\lambda) + 1]                                            \\
                        & = \frac{1}{4}[4\lambda^2 - 4\sqrt{2}\lambda + 4] = 0                                                             \\
                        & \implies \begin{cases}
                                       \lambda_1 = \frac{\sqrt{2} + \sqrt{2}i}{2} = e^{i\frac{\pi}{4}} \\
                                       \lambda_2 = \frac{\sqrt{2} - \sqrt{2}i}{2} = e^{-i\frac{\pi}{4}}
                                   \end{cases}
\end{align*}
Tìm vector riêng tương ứng với trị riêng:
\begin{align*}
    \text{Với } \lambda_1 & : (U - \lambda_1 I)\ket{u_1} = 0                       \\
                          & \implies \frac{1}{2}\begin{bmatrix}
                                                    -i(1 + \sqrt{2}) & 1               \\
                                                    -1               & i(1 - \sqrt{2})
                                                \end{bmatrix}\begin{bmatrix}
                                                                 a \\
                                                                 b
                                                             \end{bmatrix} = 0 \\
                          & \implies \begin{cases}
                                         -i(1 + \sqrt{2})a + b = 0 \\
                                         -a + i(1 - \sqrt{2})b = 0
                                     \end{cases}
    \implies \begin{cases}
                 a = 1 \\
                 b = i(1 + \sqrt{2})
             \end{cases}
    \implies \ket{u_1} = \begin{bmatrix}
                             1 \\
                             i(1 + \sqrt{2})
                         \end{bmatrix}                                            \\
    \text{Với } \lambda_2 & : (U - \lambda_2 I)\ket{u_2} = 0                       \\
                          & \implies \frac{1}{2}\begin{bmatrix}
                                                    -i(1 - \sqrt{2}) & 1               \\
                                                    -1               & i(1 + \sqrt{2})
                                                \end{bmatrix}\begin{bmatrix}
                                                                 c \\
                                                                 d
                                                             \end{bmatrix} = 0 \\
                          & \implies \begin{cases}
                                         -i(1 - \sqrt{2})c + d = 0 \\
                                         -c + i(1 + \sqrt{2})d = 0
                                     \end{cases}
    \implies \begin{cases}
                 c = 1 \\
                 d = i(1 - \sqrt{2})
             \end{cases}
    \implies \ket{u_2} = \begin{bmatrix}
                             1 \\
                             i(1 - \sqrt{2})
                         \end{bmatrix}
\end{align*}
Vậy $U$ tương ứng với phép quay quanh trục tạo bởi $\ket{u_1}, \ket{u_2}$ với góc $\theta$ trên mặt cầu Bloch, trong đó:
\begin{align*}
    \theta & = \theta_2 - \theta_1 = -\frac{\pi}{4} - \frac{\pi}{4} = -\frac{\pi}{2}
\end{align*}

\section{Bài 4}
\textbf{Đề bài:} Từ trạng thái đầu vào $|0\rangle$,

\subsection{(a)}
\textbf{Đề bài:} Vẽ mạch mô tả tính toán HYTHX.\\
\textbf{Bài làm:}
\begin{figure}[H]
    \centering
    \includegraphics[width=0.6\textwidth]{img/circuit.png}
    \caption{Mạch tính toán HYTHX}
\end{figure}

\subsection{(b)}
\textbf{Đề bài:} Tính đầu ra của Câu (a).\\
\textbf{Bài làm:}
Ta có:
\begin{align*}
    \ket{\psi} & = XHTYH\ket{0}                                                                            \\
               & = XHTY\left(\frac{1}{\sqrt{2}}\ket{0} + \frac{1}{\sqrt{2}}\ket{1}\right)                  \\
               & = XHT\left(\frac{i}{\sqrt{2}}\ket{1} - \frac{i}{\sqrt{2}}\ket{0}\right)                   \\
               & = XH\left(-\frac{i}{\sqrt{2}}\ket{0} + \frac{i}{\sqrt{2}}e^{i\frac{\pi}{4}}\ket{1}\right) \\
               & = X\left(-\frac{i}{\sqrt{2}}\ket{+} + \frac{i}{\sqrt{2}}e^{i\frac{\pi}{4}}\ket{-}\right)  \\
               & = -\frac{i}{\sqrt{2}}\ket{+} - \frac{i}{\sqrt{2}}e^{i\frac{\pi}{4}}\ket{-}
\end{align*}

\subsection{(c)}
\textbf{Đề bài:} Thêm phép đo $\tilde{\sigma}$ cuối mạch của Câu (a) và tính xác suất được 1.\\
\textbf{Bài làm:}
Ta có:
\begin{align*}
    \ket{\psi} & = -\frac{i}{\sqrt{2}}\ket{+} - \frac{i}{\sqrt{2}}e^{i\frac{\pi}{4}}\ket{-}                                                                                                                                                                                            \\
               & = \left(-\frac{i}{\sqrt{2}} \cdot \frac{1}{\sqrt{2}} - \frac{i}{\sqrt{2}}e^{i\frac{\pi}{4}} \cdot \frac{1}{\sqrt{2}}\right)\ket{0} + \left(-\frac{i}{\sqrt{2}} \cdot \frac{1}{\sqrt{2}} + \frac{i}{\sqrt{2}}e^{i\frac{\pi}{4}} \cdot \frac{1}{\sqrt{2}}\right)\ket{1} \\
               & = \left(\frac{\sqrt{2}}{4} - \frac{2+\sqrt{2}}{4}i\right)\ket{0} + \left(-\frac{\sqrt{2}}{4} + \frac{-2+\sqrt{2}}{4}i\right)\ket{1}
\end{align*}
Xác suất được $\ket{1}$ là:
$$
    \left|\left(-\frac{\sqrt{2}}{4} + \frac{-2+\sqrt{2}}{4}i\right)\right|^2 = \left(-\frac{\sqrt{2}}{4}\right)^2 + \left(\frac{-2+\sqrt{2}}{4}\right)^2 = \frac{2-\sqrt{2}}{4}
$$

\section{Bài 5}
\textbf{Đề bài:} Cho biết các trạng thái sau là tách được hay vướng, nếu tách được thì biểu diễn trên mặt cầu Bloch.

\subsection{(a)}
\textbf{Đề bài:} $\frac{1}{\sqrt{2}}(|01\rangle+|10\rangle).$\\
\textbf{Bài làm:}
Ta có:
\begin{align*}
    \ket{\phi_{1}} & = \frac{1}{\sqrt{2}}\left(\ket{01} + \ket{10}\right) = \begin{bmatrix}
                                                                                0                  \\
                                                                                \frac{1}{\sqrt{2}} \\
                                                                                \frac{1}{\sqrt{2}} \\
                                                                                0
                                                                            \end{bmatrix}
\end{align*}
Giả sử $\ket{\phi_{1}}$ tách được, tức là tồn tại $\ket{\psi_1}=\begin{bmatrix}a\\b\end{bmatrix}$ và $\ket{\psi_2}=\begin{bmatrix}c\\d\end{bmatrix}$ sao cho $\ket{\phi_{1}}=\ket{\psi_1} \otimes \ket{\psi_2}$\\
Ta có:
\begin{align*}
    \ket{\phi_{1}} & = \begin{bmatrix}
                           0                  \\
                           \frac{1}{\sqrt{2}} \\
                           \frac{1}{\sqrt{2}} \\
                           0
                       \end{bmatrix} = \begin{bmatrix}
                                           ac \\
                                           ad \\
                                           bc \\
                                           bd
                                       \end{bmatrix}
\end{align*}
Từ đó ta có hệ phương trình:
\begin{align*}
    \begin{cases}
        ac = 0                  \\
        ad = \frac{1}{\sqrt{2}} \\
        bc = \frac{1}{\sqrt{2}} \\
        bd = 0
    \end{cases}
    \implies
    \begin{cases}
        acbd = 0 \\
        acbd = \frac{1}{2}
    \end{cases}
    (\text{Mâu thuẫn})
\end{align*}
Vậy $\ket{\phi_{1}}$ không tách được.

\subsection{(b)}
\textbf{Đề bài:} $\frac{1}{\sqrt{2}}(|10\rangle+i|11\rangle)$.\\
\textbf{Bài làm:}
Ta có:
\begin{align*}
    |\phi_{2}\rangle & =\frac{1}{\sqrt{2}}\left(|10\rangle+i|11\rangle\right)
    \\
                     & = \frac{1}{\sqrt{2}}\left[|1\rangle \otimes \left(|0\rangle + i|1\rangle\right) \right]    \\
                     & = |1\rangle \otimes \left(\frac{1}{\sqrt{2}}|0\rangle + \frac{i}{\sqrt{2}}|1\rangle\right) \\
                     & = |1\rangle \otimes |i\rangle
\end{align*}
Vậy $|\phi_{2}\rangle$ tách được.\\
Dạng Bloch:
\begin{align*}
    \ket{1} & = \cos\frac{\pi}{2}\ket{0} + e^{i0}\sin\frac{\pi}{2}\ket{1} \implies \theta = \pi, \phi = 0                                   \\
    \ket{i} & = \cos\frac{\pi}{4}\ket{0} + e^{i\frac{\pi}{2}}\sin\frac{\pi}{4}\ket{1} \implies \theta = \frac{\pi}{2}, \phi = \frac{\pi}{2}
\end{align*}
Hình vẽ:
\begin{figure}[H]
    \centering
    \begin{minipage}{0.45\textwidth}
        \centering
        \includegraphics[width=0.8\textwidth]{img/5b1.png}
        \caption{Trạng thái $\ket{1}$ trên mặt cầu Bloch}
    \end{minipage}
    \hfill
    \begin{minipage}{0.45\textwidth}
        \centering
        \includegraphics[width=0.8\textwidth]{img/5b2.png}
        \caption{Trạng thái $\ket{i}$ trên mặt cầu Bloch}
    \end{minipage}
\end{figure}

\subsection{(c)}
\textbf{Đề bài:} $\frac{1}{4}(3|00\rangle-\sqrt{3}|01\rangle+\sqrt{3}|10\rangle-|11\rangle).$\\
\textbf{Bài làm:}\\
Ta có:
\begin{align*}
    \ket{\phi_{3}} & = \frac{1}{4}\left(3\ket{00}-\sqrt{3}\ket{01}+\sqrt{3}\ket{10}-\ket{11}\right)                                                          \\
                   & = \frac{1}{4}\left[\ket{0} \otimes (3\ket{0} - \sqrt{3}\ket{1}) + \frac{1}{\sqrt{3}}\ket{1} \otimes (3\ket{0} - \sqrt{3}\ket{1})\right] \\
                   & = \frac{1}{4}\left[\left(\ket{0} + \frac{1}{\sqrt{3}}\ket{1}\right) \otimes \left(3\ket{0} - \sqrt{3}\ket{1}\right)\right]              \\
                   & = \left(\frac{1}{4}\ket{0} + \frac{1}{4\sqrt{3}}\ket{1}\right) \otimes \left(3\ket{0} - \sqrt{3}\ket{1}\right)                          \\
                   & = \ket{\psi_1} \otimes \ket{\psi_2}
\end{align*}
Vậy $\ket{\phi_{3}}$ tách được.\\
Chuẩn hoá $\ket{\psi_1}$ và $\ket{\psi_2}$ và chuyển về dạng Bloch:
\begin{align*}
    \ket{\psi_1} & = \frac{\frac{1}{4}\ket{0} + \frac{1}{4\sqrt{3}}\ket{1}}{\sqrt{\left(\frac{1}{4}\right)^2 + \left(\frac{1}{4\sqrt{3}}\right)^2}} = \frac{\sqrt{3}}{2}\ket{0} + \frac{1}{2}\ket{1} = \cos\frac{\pi}{6}\ket{0} + e^{i0}\sin\frac{\pi}{6}\ket{1} \\
                 & \implies \theta = \frac{\pi}{3}, \phi = 0                                                                                                                                                                                                     \\
    \ket{\psi_2} & = \frac{3\ket{0} - \sqrt{3}\ket{1}}{\sqrt{3^2 + (-\sqrt{3})^2}} = \frac{\sqrt{3}}{2}\ket{0} - \frac{1}{2}\ket{1} = \cos\frac{\pi}{6}\ket{0} + e^{i\pi}\sin\frac{\pi}{6}\ket{1}                                                                \\
                 & \implies \theta = \frac{\pi}{3}, \phi = \pi
\end{align*}
Dạng Bloch:
\begin{align*}
    \ket{\psi_1} & = \cos\frac{\pi}{6}\ket{0} + e^{i0}\sin\frac{\pi}{6}\ket{1} \implies \theta = \frac{\pi}{3}, \phi = 0     \\
    \ket{\psi_2} & = \cos\frac{\pi}{6}\ket{0} + e^{i\pi}\sin\frac{\pi}{6}\ket{1} \implies \theta = \frac{\pi}{3}, \phi = \pi
\end{align*}
Hình vẽ:
\begin{figure}[H]
    \centering
    \begin{minipage}{0.45\textwidth}
        \centering
        \includegraphics[width=0.8\textwidth]{img/5d1.png}
        \caption{Trạng thái $\ket{\psi_1}$ trên mặt cầu Bloch}
    \end{minipage}
    \hfill
    \begin{minipage}{0.45\textwidth}
        \centering
        \includegraphics[width=0.8\textwidth]{img/5d2.png}
        \caption{Trạng thái $\ket{\psi_2}$ trên mặt cầu Bloch}
    \end{minipage}
\end{figure}

\subsection{(d)}
\textbf{Đề bài:} $\frac{1}{\sqrt{3}}|0\rangle|+\rangle+\sqrt{\frac{2}{3}}|1\rangle|-\rangle.$
\textbf{Bài làm:}\\
\begin{align*}
    |\phi_{3}\rangle & =\frac{1}{\sqrt{3}}|0+\rangle+\sqrt{\frac{2}{3}}|1-\rangle                                                       \\
                     & = \frac{1}{\sqrt{3}} \begin{bmatrix}
                                                1 \\0
                                            \end{bmatrix} \otimes \begin{bmatrix}
                                                                      \frac{1}{\sqrt{2}} \\ \frac{1}{\sqrt{2}}
                                                                  \end{bmatrix}
    + \sqrt{\frac{2}{3}} \begin{bmatrix}
                             0 \\1
                         \end{bmatrix} \otimes \begin{bmatrix}
                                                   \frac{1}{\sqrt{2}} \\ -\frac{1}{\sqrt{2}}
                                               \end{bmatrix}
    = \begin{bmatrix}
          \frac{1}{\sqrt{6}} \\
          \frac{1}{\sqrt{6}} \\
          \frac{1}{\sqrt{3}} \\
          -\frac{1}{\sqrt{3}}
      \end{bmatrix}
\end{align*}
Giả sử $|\phi_{3}\rangle$ tách được, tức là tồn tại $|\psi_1\rangle=\begin{bmatrix}a\\b\end{bmatrix}$ và $|\psi_2\rangle=\begin{bmatrix}c\\d\end{bmatrix}$ sao cho $|\phi_{4}\rangle=|\psi_1\rangle \otimes |\psi_2\rangle$\\
Ta có:
\begin{align*}
    |\psi_1\rangle \otimes |\psi_2\rangle = \begin{bmatrix}
                                                ac \\
                                                ad \\
                                                bc \\
                                                bd
                                            \end{bmatrix} \\
    |\phi_{3}\rangle=|\psi_1\rangle \otimes |\psi_2\rangle
    \implies \begin{bmatrix}
                 \frac{1}{\sqrt{6}} \\
                 \frac{1}{\sqrt{6}} \\
                 \frac{1}{\sqrt{3}} \\
                 -\frac{1}{\sqrt{3}}
             \end{bmatrix} = \begin{bmatrix}
                                 ac \\
                                 ad \\
                                 bc \\
                                 bd
                             \end{bmatrix}
\end{align*}
Từ đó ta có hệ phương trình:
\begin{align*}
    \begin{cases}
        ac = \frac{1}{\sqrt{6}} \\
        ad = \frac{1}{\sqrt{6}} \\
        bc = \frac{1}{\sqrt{3}} \\
        bd = -\frac{1}{\sqrt{3}}
    \end{cases}
    \implies
    \begin{cases}
        \frac{ac}{ad} = \frac{c}{d} = 1 \\
        \frac{bc}{bd} = \frac{c}{d} = -1
    \end{cases}
    (\text{Mâu thuẫn})
\end{align*}
Vậy $|\phi_{3}\rangle$ không tách được.

\section{Bài 6}
\textbf{Đề bài:} Cho hệ 2 qubit với trạng thái $|\psi\rangle=\frac{1}{2}|00\rangle-\frac{i}{2}|10\rangle+\frac{1}{\sqrt{2}}|11\rangle$.
Khảo sát các phép đo sau.

\subsection{(a)}
\textbf{Đề bài:} Đo đồng thời 2 qubit.\\
\textbf{Bài làm:}\\
Khi đo đồng thời 2 qubit theo cơ sở tính toán, ta có các kết quả và xác suất tương ứng như sau:
\begin{itemize}
    \item Được $\ket{00}$ với xác suất $P_{00} = \left|\frac{1}{2}\right|^2 = \frac{1}{4}$
    \item Được $\ket{10}$ với xác suất $P_{10} = \left|-\frac{i}{2}\right|^2 = \frac{1}{4}$
    \item Được $\ket{11}$ với xác suất $P_{11} = \left|\frac{1}{\sqrt{2}}\right|^2 = \frac{1}{2}$
\end{itemize}

\subsection{(b)}
\textbf{Đề bài:} Đo qubit 0.\\
\textbf{Bài làm:}\\
Để đo riêng qubit 0 theo cơ sở tính toán, ta viết lại:
\begin{align*}
    \ket{\psi} & = \frac{1}{2}\ket{0}\otimes\ket{0} - \frac{i}{2}\ket{1}\otimes\ket{0} + \frac{1}{\sqrt{2}}\ket{1}\otimes\ket{1}  \\
               & = \left(\frac{1}{2}\ket{0} - \frac{i}{2}\ket{1}\right) \otimes \ket{0} + \frac{1}{\sqrt{2}}\ket{1}\otimes\ket{1}
\end{align*}
Khi đó, qubit 0 được:
\begin{itemize}
    \item $\ket{0}$ với xác suất $\left|\frac{1}{2}\ket{0} - \frac{i}{2}\ket{1}\right|^2 = \frac{1}{2}$.
    \item $\ket{1}$ với xác suất $\left|\frac{1}{\sqrt{2}}\ket{1}\right|^2 = \frac{1}{2}$.
\end{itemize}

\subsection{(c)}
\textbf{Đề bài:} Đo qubit 1.\\
\textbf{Bài làm:}\\
Để đo riêng qubit 1 theo cơ sở tính toán, ta viết lại:
\begin{align*}
    \ket{\psi} & = \frac{1}{2}\ket{0}\otimes\ket{0} - \frac{i}{2}\ket{1}\otimes\ket{0} + \frac{1}{\sqrt{2}}\ket{1}\otimes\ket{1}   \\
               & = \ket{0} \otimes \frac{1}{2}\ket{0} + \ket{1}\otimes\left(-\frac{i}{2}\ket{0} + \frac{1}{\sqrt{2}}\ket{1}\right)
\end{align*}
Khi đó, qubit 1 được:
\begin{itemize}
    \item $\ket{0}$ với xác suất $\left|\frac{1}{2}\ket{0}\right|^2 = \frac{1}{4}$.
    \item $\ket{1}$ với xác suất $\left|-\frac{i}{2}\ket{0} + \frac{1}{\sqrt{2}}\ket{1}\right|^2 = \frac{3}{4}$.
\end{itemize}

\subsection{(d)}
\textbf{Đề bài:} Đo qubit 0 rồi đo qubit 1 và so kết quả với Câu (a).\\
\textbf{Bài làm:}\\
Khi đo qubit 0, ta có 2 trường hợp:
\begin{itemize}
    \item Qubit 0 được $\ket{0}$ với xác suất $\frac{1}{2}$ và hệ sụp đổ thành:
          $$
              \ket{\psi_0} = \frac{\frac{1}{2}\ket{0} - \frac{i}{2}\ket{1}}{\left|\frac{1}{2}\ket{0} - \frac{i}{2}\ket{1}\right|} \otimes \ket{0} = \left(\frac{1}{\sqrt{2}}\ket{0} - \frac{i}{\sqrt{2}}\ket{1}\right) \otimes \ket{0}
          $$
          Tiếp tục đo qubit 1, ta được $\ket{00}$ với xác suất $\frac{1}{2}$, được $\ket{10}$ với xác suất $\frac{1}{2}$.
    \item Qubit 0 được $\ket{1}$ với xác suất $\frac{1}{2}$ và hệ sụp đổ thành:
          $$
              \ket{\psi_0} = \frac{\frac{1}{\sqrt{2}}\ket{1}}{\left|\frac{1}{\sqrt{2}}\ket{1}\right|} \otimes \ket{1} = \ket{1} \otimes \ket{1}
          $$
          Tiếp tục đo qubit 1, ta được $\ket{11}$ với xác suất $1$.
\end{itemize}
Tổng hợp lại, ta có các kết quả và xác suất tương ứng như sau:
\begin{itemize}
    \item Được $\ket{00}$ với xác suất $P_{00} = \frac{1}{2} \cdot \frac{1}{2} = \frac{1}{4}$
    \item Được $\ket{10}$ với xác suất $P_{10} = \frac{1}{2} \cdot \frac{1}{2}= \frac{1}{4}$
    \item Được $\ket{11}$ với xác suất $P_{11} = \frac{1}{2} \cdot 1 = \frac{1}{2}$
\end{itemize}
So sánh với Câu (a), ta thấy kết quả giống hệt nhau. Vậy việc đo qubit 0 trước không làm thay đổi xác suất thu được các trạng thái khi đo đồng thời 2 qubit.



\subsection{(e)}
\textbf{Đề bài:} Đo qubit 1 rồi đo qubit 0 và so kết quả với Câu (a).\\
\textbf{Bài làm:}\\
Khi đo qubit 1, ta có 2 trường hợp:
\begin{itemize}
    \item Qubit 1 được $\ket{0}$ với xác suất $\frac{1}{4}$ và hệ sụp đổ thành:
          $$
              \ket{\psi_1} = \frac{\frac{1}{2}\ket{0}}{\left|\frac{1}{2}\ket{0}\right|} \otimes \ket{0} = \ket{0} \otimes \ket{0}
          $$
          Tiếp tục đo qubit 0, ta được $\ket{00}$ với xác suất $1$.
    \item Qubit 1 được $\ket{1}$ với xác suất $\frac{3}{4}$ và hệ sụp đổ thành:
          $$
              \ket{\psi_1} = \frac{-\frac{i}{2}\ket{1} + \frac{1}{\sqrt{2}}\ket{1}}{\left|-\frac{i}{2}\ket{1} + \frac{1}{\sqrt{2}}\ket{1}\right|} \otimes \ket{1} = \left(-\frac{i}{\sqrt{3}}\ket{0} + \sqrt{\frac{2}{3}}\ket{1}\right) \otimes \ket{1}
          $$
          Tiếp tục đo qubit 0, ta được $\ket{10}$ với xác suất $\frac{1}{3}$, được $\ket{11}$ với xác suất $\frac{2}{3}$.
\end{itemize}
Tổng hợp lại, ta có các kết quả và xác suất tương ứng như sau:
\begin{itemize}
    \item Được $\ket{00}$ với xác suất $P_{00} = \frac{1}{4} \cdot 1 = \frac{1}{4}$
    \item Được $\ket{10}$ với xác suất $P_{10} = \frac{3}{4} \cdot \frac{1}{3} = \frac{1}{4}$
    \item Được $\ket{11}$ với xác suất $P_{11} = \frac{3}{4} \cdot \frac{2}{3} = \frac{1}{2}$
\end{itemize}
So sánh với Câu (a), ta thấy kết quả giống hệt nhau. Vậy việc đo qubit 1 trước không làm thay đổi xác suất thu được các trạng thái khi đo đồng thời 2 qubit.

\section{Bài 7}
\textbf{Đề bài:} Khảo sát phép toán 2 qubit $U=H\otimes X$.

\subsection{(a)}
\textbf{Đề bài:} Cho biết tác động của U lên các vector của cơ sở tính toán.\\
\textbf{Bài làm:}
\begin{align*}
    U\ket{00} & = (H\otimes X)(\ket{0}\otimes\ket{0}) = H\ket{0} \otimes X\ket{0} = \frac{1}{\sqrt{2}}(\ket{0} + \ket{1}) \otimes \ket{1} = \frac{1}{\sqrt{2}}\ket{01} + \frac{1}{\sqrt{2}}\ket{11} \\
    U\ket{01} & = (H\otimes X)(\ket{0}\otimes\ket{1}) = H\ket{0} \otimes X\ket{1} = \frac{1}{\sqrt{2}}(\ket{0} + \ket{1}) \otimes \ket{0} = \frac{1}{\sqrt{2}}\ket{00} + \frac{1}{\sqrt{2}}\ket{10} \\
    U\ket{10} & = (H\otimes X)(\ket{1}\otimes\ket{0}) = H\ket{1} \otimes X\ket{0} = \frac{1}{\sqrt{2}}(\ket{0} - \ket{1}) \otimes \ket{1} = \frac{1}{\sqrt{2}}\ket{01} - \frac{1}{\sqrt{2}}\ket{11} \\
    U\ket{11} & = (H\otimes X)(\ket{1}\otimes\ket{1}) = H\ket{1} \otimes X\ket{1} = \frac{1}{\sqrt{2}}(\ket{0} - \ket{1}) \otimes \ket{0} = \frac{1}{\sqrt{2}}\ket{00} - \frac{1}{\sqrt{2}}\ket{10}
\end{align*}

\subsection{(b)}
\textbf{Đề bài:} Xác định ma trận biểu diễn của U từ Câu (a).\\
\textbf{Bài làm:}\\
Từ Câu (a), ta có ma trận biểu diễn của $U$ là:
\begin{align*}
    U & = \begin{bmatrix}
              U\ket{00} & U\ket{01} & U\ket{10} & U\ket{11}
          \end{bmatrix}
    = \frac{1}{\sqrt{2}}\begin{bmatrix}
                            0 & 1 & 0  & 1  \\
                            1 & 0 & 1  & 0  \\
                            0 & 1 & 0  & -1 \\
                            1 & 0 & -1 & 0
                        \end{bmatrix}
\end{align*}

\subsection{(c)}
\textbf{Đề bài:} Xác định ma trận biểu diễn của U bằng phép tích tensor.\\
\textbf{Bài làm:}\\
\begin{align*}
    H =                   & \frac{1}{\sqrt{2}}\begin{bmatrix}
                                                  1 & 1  \\
                                                  1 & -1
                                              \end{bmatrix}, \quad
    X = \begin{bmatrix}
            0 & 1 \\
            1 & 0
        \end{bmatrix}                                                                                       \\
    \implies H\otimes X = & \frac{1}{\sqrt{2}}\begin{bmatrix}
                                                  1 & 1  \\
                                                  1 & -1
                                              \end{bmatrix} \otimes \begin{bmatrix}
                                                                        0 & 1 \\
                                                                        1 & 0
                                                                    \end{bmatrix}                           \\
    =                     & \frac{1}{\sqrt{2}}\begin{bmatrix}
                                                  1 \begin{bmatrix} 0 & 1 \\ 1 & 0 \end{bmatrix} & 1 \begin{bmatrix} 0 & 1 \\ 1 & 0 \end{bmatrix}  \\
                                                  1 \begin{bmatrix} 0 & 1 \\ 1 & 0 \end{bmatrix} & -1 \begin{bmatrix} 0 & 1 \\ 1 & 0 \end{bmatrix}
                                              \end{bmatrix} \\
    =                     & \frac{1}{\sqrt{2}}\begin{bmatrix}
                                                  0 & 1 & 0  & 1  \\
                                                  1 & 0 & 1  & 0  \\
                                                  0 & 1 & 0  & -1 \\
                                                  1 & 0 & -1 & 0
                                              \end{bmatrix}
\end{align*}

\subsection{(d)}
\textbf{Đề bài:} Cho biết tác động của U lên trạng thái $|\psi\rangle=\frac{1}{4}|00\rangle+\frac{1}{2}|01\rangle+\frac{1}{\sqrt{2}}|10\rangle+\frac{\sqrt{3}}{4}|11\rangle$.\\
\textbf{Bài làm:}
\begin{align*}
    U\ket{\psi} & = U\left(\frac{1}{4}\ket{00} + \frac{1}{2}\ket{01} + \frac{1}{\sqrt{2}}\ket{10} + \frac{\sqrt{3}}{4}\ket{11}\right)                                                                                                                                               \\
                & = \frac{1}{4}U\ket{00} + \frac{1}{2}U\ket{01} + \frac{1}{\sqrt{2}}U\ket{10} + \frac{\sqrt{3}}{4}U\ket{11}                                                                                                                                                         \\
                & = \frac{1}{4}\left(\frac{1}{\sqrt{2}}\ket{01} + \frac{1}{\sqrt{2}}\ket{11}\right) + \frac{1}{2}\left(\frac{1}{\sqrt{2}}\ket{00} + \frac{1}{\sqrt{2}}\ket{10}\right)                                                                                               \\
                & \quad + \frac{1}{\sqrt{2}}\left(\frac{1}{\sqrt{2}}\ket{01} - \frac{1}{\sqrt{2}}\ket{11}\right) + \frac{\sqrt{3}}{4}\left(\frac{1}{\sqrt{2}}\ket{00} - \frac{1}{\sqrt{2}}\ket{10}\right)                                                                           \\
                & = \left(\frac{1}{2\sqrt{2}} + \frac{\sqrt{3}}{4\sqrt{2}}\right)\ket{00} + \left(\frac{1}{4\sqrt{2}} + \frac{1}{2}\right)\ket{01} + \left(\frac{1}{2\sqrt{2}} - \frac{\sqrt{3}}{4\sqrt{2}}\right)\ket{10} + \left(\frac{1}{4\sqrt{2}} - \frac{1}{2}\right)\ket{11}
\end{align*}

\section{Bài 8}
\textbf{Đề bài:} Xét trạng thái 3 qubit $|GHZ\rangle=\frac{1}{\sqrt{2}}|000\rangle+\frac{1}{\sqrt{2}}|111\rangle$.

\subsection{(a)}
\textbf{Đề bài:} Chứng minh $|GHZ\rangle$ là trạng thái vướng.\\
\textbf{Bài làm:}\\
Giả sử $\ket{GHZ}$ tách được, tức là tồn tại $\ket{\psi_1}=\begin{bmatrix}a\\b\end{bmatrix}$, $\ket{\psi_2}=\begin{bmatrix}c\\d\end{bmatrix}$ và $\ket{\psi_3}=\begin{bmatrix}e\\f\end{bmatrix}$ sao cho $\ket{GHZ}=\ket{\psi_1} \otimes \ket{\psi_2} \otimes \ket{\psi_3}$\\
Ta có:
\begin{align*}
    \ket{GHZ} & = \frac{1}{\sqrt{2}}\left(\ket{000} + \ket{111}\right) = \frac{1}{\sqrt{2}}\begin{bmatrix}
                                                                                               1 \\0 \\0 \\0 \\0 \\0 \\0 \\1
                                                                                           \end{bmatrix} = \begin{bmatrix}
                                                                                                               \frac{1}{\sqrt{2}} \\0 \\0 \\0 \\0 \\0 \\0 \\ \frac{1}{\sqrt{2}}
                                                                                                           \end{bmatrix}
    = \begin{bmatrix}
          ace \\
          acf \\
          ade \\
          adf \\
          bce \\
          bcf \\
          bde \\
          bdf
      \end{bmatrix}
\end{align*}
Từ đó ta có hệ phương trình:
\begin{align*}
    \begin{cases}
        ace = \frac{1}{\sqrt{2}} \\
        acf = 0                  \\
        ade = 0                  \\
        adf = 0                  \\
        bce = 0                  \\
        bcf = 0                  \\
        bde = 0                  \\
        bdf = \frac{1}{\sqrt{2}}
    \end{cases}
    \implies
    \begin{cases}
        ace bdf = 0 \\
        ace bdf = \frac{1}{2}
    \end{cases}
    (\text{Mâu thuẫn})
\end{align*}
Vậy $\ket{GHZ}$ không tách được.

\subsection{(b)}
\textbf{Đề bài:} Khảo sát phép đo riêng qubit 0, qubit 1, qubit 2 và nhận xét.\\
\textbf{Bài làm:}\\
Để đo riêng qubit 0 theo cơ sở tính toán, ta viết lại:
\begin{align*}
    \ket{GHZ} & = \frac{1}{\sqrt{2}}\ket{00}\otimes\ket{0} + \frac{1}{\sqrt{2}}\ket{11}\otimes\ket{1}
\end{align*}
Khi đó, qubit 0 được:
\begin{itemize}
    \item $\ket{0}$ với xác suất $\left|\frac{1}{\sqrt{2}}\ket{00}\right|^2 = \frac{1}{2}$.
    \item $\ket{1}$ với xác suất $\left|\frac{1}{\sqrt{2}}\ket{11}\right|^2 = \frac{1}{2}$.
\end{itemize}
Tương tự, khi đo riêng qubit 1 và qubit 2 theo cơ sở tính toán, ta cũng thu được kết quả:
\begin{itemize}
    \item Được $\ket{0}$ với xác suất $\frac{1}{2}$.
    \item Được $\ket{1}$ với xác suất $\frac{1}{2}$.
\end{itemize}
Nhận xét: Khi đo riêng lẻ, mỗi qubit đều có xác suất $50\%$ để thu được trạng thái $\ket{0}$ hoặc $\ket{1}$, cho dù toàn bộ hệ thống ở trong trạng thái rối lượng tử $\ket{GHZ}$. Ngoài ra, vì $\ket{GHZ}$ là trạng thái vướng, nên khi đo riêng 1 qubit sẽ làm sụp đổ toàn bộ trạng thái của hệ về $\ket{000}$ hoặc $\ket{111}$, khi đó các qubit còn lại sẽ có xác suất $100\%$ để thu được trạng thái tương ứng.

\subsection{(c)}
\textbf{Đề bài:} Thiết kế mạch 3 qubit để tạo trạng thái $|GHZ\rangle$.\\
\textbf{Bài làm:}\\
Để tạo trạng thái $|GHZ\rangle$, ta có thể sử dụng mạch lượng tử gồm các cổng Hadamard (H) và cổng CNOT như sau:
\begin{align*}
    \ket{000} & \xrightarrow{H \text{ trên qubit 0}} \frac{1}{\sqrt{2}}(\ket{000} + \ket{001})                                \\
              & \xrightarrow{CNOT \text{ (qubit 0 điều khiển qubit 1)}} \frac{1}{\sqrt{2}}(\ket{000} + \ket{011})             \\
              & \xrightarrow{CNOT \text{ (qubit 0 điều khiển qubit 2)}} \frac{1}{\sqrt{2}}(\ket{000} + \ket{111}) = \ket{GHZ}
\end{align*}

Hình vẽ:
\begin{figure}[H]
    \centering
    \includegraphics[width=0.6\textwidth]{img/ghz.png}
    \caption{Mạch lượng tử tạo trạng thái $|GHZ\rangle$}
\end{figure}