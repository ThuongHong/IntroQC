\section{Bài 1}
Cho $x=e^{i\frac{\pi}{3}}$ và $y=2e^{i\frac{\pi}{6}}$.

\subsection{(a)}
\textbf{Đề bài:} Vẽ minh họa $x, y$ trên mặt phẳng phức.\\
\textbf{Bài làm:}
\begin{center}
\begin{tikzpicture}[scale=2]

    % gốc tọa độ và trục
    \draw[->, gray] (-0.5, 0) -- (2.5, 0) node[right] {$\mathrm{Re}$};
    \draw[->, gray] (0, -0.5) -- (0, 1.5) node[above] {$\mathrm{Im}$};
    \coordinate (O) at (0, 0);

    % số phức x
    \coordinate (X) at ({cos(60)}, {sin(60)});
    \draw[->, very thick, blue!70] (O) -- (X) node[above right] {$x$};
    \draw[blue!70] (0.3, 0) arc (0:60:0.3);

    % số phức y
    \coordinate (Y) at ({2*cos(30)}, {2*sin(30)});
    \draw[->, very thick, red!70] (O) -- (Y) node[above] {$y$};
    \draw[red!70] (0.5, 0) arc (0:30:0.5);

\end{tikzpicture}
\end{center}

\subsection{(b)}
\textbf{Đề bài:} Tìm dạng đại số và dạng cực của $x, y$.\\
\textbf{Bài làm:}
\begin{itemize}
    \item \textbf{Dạng cực}:
        \begin{align*}
            x &= \cos\left(\frac{\pi}{3}\right) + i\sin\left(\frac{\pi}{3}\right)\\
            y &= 2\left(\cos\left(\frac{\pi}{6}\right) + i\sin\left(\frac{\pi}{6}\right)\right)
        \end{align*}

    \item \textbf{Dạng đại số}:
        \begin{align*}
            x &= \frac{1}{2} + i\frac{\sqrt{3}}{2}\\
            y &= \sqrt{3} + i
        \end{align*}
\end{itemize}

\subsection{(c)}
\textbf{Đề bài:} Tính $\mathrm{Re}(x)$, $\mathrm{Im}(x)$, $|x|$, $\arg(x)$.\\
\textbf{Bài làm:}
\begin{itemize}
    \item $\mathrm{Re}(x) = \frac{1}{2}$
    \item $\mathrm{Im}(x) = \frac{\sqrt{3}}{2}$
    \item $|x| = 1$
    \item $\arg(x) = \frac{\pi}{3}$
\end{itemize}

\subsection{(d)}
\textbf{Đề bài:} Tính $\overline{x},-x,x^{-1}$.\\
\textbf{Bài làm:}
\begin{itemize}
    \item $\overline{x} = \frac{1}{2} - i\frac{\sqrt{3}}{2}$
    \item $-x = -\frac{1}{2} - i\frac{\sqrt{3}}{2}$
    \item $x^{-1} = \frac{1}{x} = \frac{1}{2} - i\frac{\sqrt{3}}{2}$
\end{itemize}

\subsection{(e)}
\textbf{Đề bài:} Tính $x + y, x - y, xy, \frac{x}{y}, \frac{y}{x}$.\\
\textbf{Bài làm:}
\begin{itemize}
    \item $x + y = \frac{1}{2} + i\frac{\sqrt{3}}{2} + \sqrt{3} + i = \left(\frac{1}{2} + \sqrt{3}\right) + i\left(\frac{\sqrt{3}}{2} + 1\right)$
    \item $x - y = \frac{1}{2} + i\frac{\sqrt{3}}{2} - \left(\sqrt{3} + i\right) = \left(\frac{1}{2} - \sqrt{3}\right) + i\left(\frac{\sqrt{3}}{2} - 1\right)$
    \item $xy = e^{i\frac{\pi}{3}} \cdot 2e^{i\frac{\pi}{6}} = 2e^{i\left(\frac{\pi}{3} + \frac{\pi}{6}\right)} = 2e^{i\frac{\pi}{2}} = 2i$
    \item $\frac{x}{y} = \frac{e^{i\frac{\pi}{3}}}{2e^{i\frac{\pi}{6}}} = \frac{1}{2}e^{i\left(\frac{\pi}{3} - \frac{\pi}{6}\right)} = \frac{1}{2}e^{i\frac{\pi}{6}} = \frac{\sqrt{3}}{4} + i\frac{1}{4}$
    \item $\frac{y}{x} = \frac{2e^{i\frac{\pi}{6}}}{e^{i\frac{\pi}{3}}} = 2e^{i\left(\frac{\pi}{6} - \frac{\pi}{3}\right)} = 2e^{-i\frac{\pi}{6}} = \sqrt{3} - i$
\end{itemize}

\subsection{(f)}
\textbf{Đề bài:} Tính $x^{4}$ và $x^{n}$, $n\in\mathbb{Z}$.\\
\textbf{Bài làm:}
\begin{itemize}
    \item $x^{4} = \left(e^{i\frac{\pi}{3}}\right)^4 = e^{i\frac{4\pi}{3}}$
    \item $x^{n} = \left(e^{i\frac{\pi}{3}}\right)^{n} = e^{i\frac{n\pi}{3}}$
\end{itemize}

\subsection{(g)}
\textbf{Đề bài:} Tính $\sqrt[4]{x}$ và $\sqrt[n]{x}$ $n\in\mathbb{N}^{+}$.\\
\textbf{Bài làm:}
\begin{itemize}
    \item $\sqrt[4]{x} = \left(e^{i\frac{\pi}{3}}\right)^{\frac{1}{4}} = e^{i\frac{\pi}{12}}$
    \item $\sqrt[n]{x} = \left(e^{i\frac{\pi}{3}}\right)^{\frac{1}{n}} = e^{i\frac{\pi}{3n}}$
\end{itemize}

\section{Bài 2}
Cho $x, y\in\mathbb{C}$ chứng minh

\subsection{(a)}
\textbf{Đề bài:} $x\overline{x}=\overline{x}x=|x|^{2}$.\\
\textbf{Bài làm:}
\begin{align*}
    x \overline{x} &= (a + bi)(a - bi) = a^{2} + b^{2} = |x|^{2} \quad (\text{ĐPCM})\\
\end{align*}

\subsection{(b)}
\textbf{Đề bài:} $\overline{x^{-1}}=(\overline{x})^{-1}(x\ne0)$.\\
\textbf{Bài làm:}
\begin{align*}
    \overline{x^{-1}} &= \overline{\left(\frac{1}{x}\right)} = \frac{1}{\overline{x}} = (\overline{x})^{-1} \quad (\text{ĐPCM})\\
\end{align*}

\subsection{(c)}
\textbf{Đề bài:} $|xy|=|x||y|$.\\
\textbf{Bài làm:}
\\Xét $x=a+bi$, $y=c+di$. \\
Ta có:
\begin{align}
    |xy| &= |(a + bi)(c + di)| = |(ac - bd) + (ad + bc)i| = \sqrt{(ac - bd)^{2} + (ad + bc)^{2}}\\
    &= \sqrt{a^{2}c^{2} - 2abcd + b^{2}d^{2} + a^{2}d^{2} + 2abcd + b^{2}c^{2}} = \sqrt{a^{2}c^{2} + b^{2}d^{2} + a^{2}d^{2} + b^{2}c^{2}}\\
\end{align}
Mặt khác:
\begin{align}
    |x||y| &= \sqrt{a^{2} + b^{2}}\sqrt{c^{2} + d^{2}} = \sqrt{(a^{2} + b^{2})(c^{2} + d^{2})} = \sqrt{a^{2}c^{2} + a^{2}d^{2} + b^{2}c^{2} + b^{2}d^{2}}\\ 
\end{align}
Vậy $|xy| = |x||y|$ (ĐPCM).

\subsection{(d)}
\textbf{Đề bài:} $|x+y|\le|x|+|y|$.\\
\textbf{Bài làm:}\\
Xét $x=a+bi$, $y=c+di$. \\
Ta có:
\begin{align*}
    |x+y| &= |(a+c) + (b+d)i| = \sqrt{(a+c)^{2} + (b+d)^{2}} = \sqrt{a^{2} + 2ac + c^{2} + b^{2} + 2bd + d^{2}}\\
    &= \sqrt{(a^{2} + b^{2}) + (c^{2} + d^{2}) + 2(ac + bd)}\\
    &\le \sqrt{(a^{2} + b^{2}) + (c^{2} + d^{2}) + 2\sqrt{(a^{2} + b^{2})(c^{2} + d^{2})}} \quad (\text{BĐT Cauchy-Schwarz})\\
    &= \sqrt{(\sqrt{a^{2} + b^{2}} + \sqrt{c^{2} + d^{2}})^{2}} = \sqrt{(|x| + |y|)^2} =|x| + |y| \quad (\text{ĐPCM})
\end{align*}

\section{Bài 3}
Cho $|\phi\rangle=\frac{\sqrt{3}}{2}|0\rangle+\frac{1}{2}|1\rangle$, $|\psi\rangle=\frac{2}{3}|0\rangle+\frac{1-2i}{3}|1\rangle$.

\subsection{(a)}
\textbf{Đề bài:} Tính $\langle\phi|$ và $\langle\psi|$.\\
\textbf{Bài làm:}
\begin{itemize}
    \item $\langle\phi| = \dots$
    \item $\langle\psi| = \dots$
\end{itemize}

\subsection{(b)}
\textbf{Đề bài:} Tính $\langle\phi|\psi\rangle$ và $\langle\psi|\phi\rangle$.\\
\textbf{Bài làm:}
\begin{itemize}
    \item $\langle\phi|\psi\rangle = \dots$
    \item $\langle\psi|\phi\rangle = \dots$
\end{itemize}

\subsection{(c)}
\textbf{Đề bài:} Tính $|\phi\rangle\langle\phi|$ và $|\psi\rangle\langle\phi|$.\\
\textbf{Bài làm:}
\begin{itemize}
    \item $|\phi\rangle\langle\phi| = \dots$
    \item $|\psi\rangle\langle\phi| = \dots$
\end{itemize}

\subsection{(d)}
\textbf{Đề bài:} Tính $|\phi\rangle|\psi\rangle$ và $|\psi\rangle|\phi\rangle$.\\
\textbf{Bài làm:} (Tensor product)
\begin{itemize}
    \item $|\phi\rangle|\psi\rangle = \dots$
    \item $|\psi\rangle|\phi\rangle = \dots$
\end{itemize}

\subsection{(e)}
\textbf{Đề bài:} Tính $||\phi||$ và $||\psi||$.\\
\textbf{Bài làm:}
\begin{itemize}
    \item $||\phi|| = \dots$
    \item $||\psi|| = \dots$
\end{itemize}

\subsection{(f)}
\textbf{Đề bài:} Tính góc giữa $|\phi\rangle$ và $|\psi\rangle$.\\
\textbf{Bài làm:}

\subsection{(g)}
\textbf{Đề bài:} Tính $\mathrm{proj}_{|\psi\rangle}|\phi\rangle$ và $\mathrm{proj}_{|\phi\rangle}|\psi\rangle$.\\
\textbf{Bài làm:}
\begin{itemize}
    \item $\mathrm{proj}_{|\psi\rangle}|\phi\rangle = \dots$
    \item $\mathrm{proj}_{|\phi\rangle}|\psi\rangle = \dots$
\end{itemize}

\subsection{(h)}
\textbf{Đề bài:} Chuẩn hóa $\mathrm{proj}_{|\psi\rangle}|\phi\rangle$ và $\mathrm{proj}_{|\phi\rangle}|\psi\rangle$.\\
\textbf{Bài làm:}
\begin{itemize}
    \item Chuẩn hóa $\mathrm{proj}_{|\psi\rangle}|\phi\rangle = \dots$
    \item Chuẩn hóa $\mathrm{proj}_{|\phi\rangle}|\psi\rangle = \dots$
\end{itemize}

\subsection{(i)}
\textbf{Đề bài:} Tìm tọa độ của $|\phi\rangle$ và $|\psi\rangle$ trong các cơ sở $B_{Z}=\{|0\rangle,|1\rangle\}$, $B_{X}=\{|+\rangle,|-\rangle\}$, $B_{Y}=\{|i\rangle,|-i\rangle\}$.\\
\textbf{Bài làm:}

\subsection{(j)}
\textbf{Đề bài:} Cho $|a\rangle=\frac{\sqrt{3}}{2}|0\rangle+\frac{i}{2}|1\rangle$, $|b\rangle=\frac{i}{2}|0\rangle+\frac{\sqrt{3}}{2}|1\rangle$, chứng minh $B=\{a,b\}$ là một cơ sở trực chuẩn của $\mathbb{C}^{2}$ và tìm tọa độ của $|\phi\rangle$, $|\psi\rangle$ theo $B$.
\textbf{Bài làm:}

\section{Bài 4}
Cho $U$ là toán tử trên $\mathbb{C}^{2}$ với $U|0\rangle=\frac{1}{\sqrt{2}}\begin{pmatrix}1\\ -i\end{pmatrix}$ và $U|1\rangle=\frac{1}{\sqrt{2}}\begin{pmatrix}-i\\ 1\end{pmatrix}$.

\subsection{(a)}
\textbf{Đề bài:} Tìm biểu diễn của $U$ trong cơ sở chính tắc $B_{Z}=\{|0\rangle,|1\rangle\}$.
\textbf{Bài làm:}

\subsection{(b)}
\textbf{Đề bài:} Cho $|\phi\rangle=\begin{pmatrix}\alpha\\ \beta\end{pmatrix}\in\mathbb{C}^{2},$ tìm $U|\phi\rangle$.
\textbf{Bài làm:}

\subsection{(c)}
\textbf{Đề bài:} $U$ có unita không?
\textbf{Bài làm:}

\subsection{(d)}
\textbf{Đề bài:} $U$ có Hermite không?
\textbf{Bài làm:}

\subsection{(e)}
\textbf{Đề bài:} Tìm $U^{\dagger}$, $U^{-1}$.
\textbf{Bài làm:}
\begin{itemize}
    \item $U^{\dagger} = \dots$
    \item $U^{-1} = \dots$
\end{itemize}

\subsection{(f)}
\textbf{Đề bài:} Tìm $HUH|0\rangle$, $HUH|1\rangle$ và $HUH$ ($H$ là ma trận Hadamard).
\textbf{Bài làm:}

\subsection{(g)}
\textbf{Đề bài:} Tìm $UHU|0\rangle$, $UHU|1\rangle$ và $UHU$.
\textbf{Bài làm:}

\section{Bài 5}
Chứng minh $XY=iZ$ bằng cách

\subsection{(a)}
\textbf{Đề bài:} Nhân ma trận.
\textbf{Bài làm:}

\subsection{(b)}
\textbf{Đề bài:} Xét tác động của các toán tử trên $|0\rangle$, $|1\rangle$.
\textbf{Bài làm:}

\section{Bài 6}
Cho $|\phi\rangle=\frac{1}{2}|00\rangle+\frac{i}{\sqrt{2}}|10\rangle+\frac{\sqrt{3}+i}{4}|11\rangle$.

\subsection{(a)}
\textbf{Đề bài:} Cho thấy $|\phi\rangle$ là vector đơn vị.
\textbf{Bài làm:}

\subsection{(b)}
\textbf{Đề bài:} Tính $\mathrm{proj}_{|+-\rangle}|\phi\rangle$ và chuẩn hóa $\mathrm{proj}_{|+-\rangle}|\phi\rangle$.
\textbf{Bài làm:}
\begin{itemize}
    \item $\mathrm{proj}_{|+-\rangle}|\phi\rangle = \dots$
    \item Chuẩn hóa $\mathrm{proj}_{|+-\rangle}|\phi\rangle = \dots$
\end{itemize}

\subsection{(c)}
\textbf{Đề bài:} Tính tọa độ của $|\phi\rangle$ theo cơ sở Bell.
\textbf{Bài làm:}

\section{Bài 7}
Kiểm tra các vector sau có phân tách được (separable)

\subsection{(a)}
\textbf{Đề bài:} $|\phi_{1}\rangle=\frac{1}{2}(|00\rangle-|01\rangle+|10\rangle-|11\rangle).$
\textbf{Bài làm:}

\subsection{(b)}
\textbf{Đề bài:} $|\phi_{2}\rangle=\frac{1}{2\sqrt{2}}(\sqrt{3}|00\rangle-\sqrt{3}|01\rangle+|10\rangle-|11\rangle)$.
\textbf{Bài làm:}

\subsection{(c)}
\textbf{Đề bài:} $|\phi_{3}\rangle=\frac{1}{\sqrt{2}}(|10\rangle+i|11\rangle)$.
\textbf{Bài làm:}

\subsection{(d)}
\textbf{Đề bài:} $|\phi_{4}\rangle=\frac{1}{\sqrt{3}}|0+\rangle+\sqrt{\frac{2}{3}}|1-\rangle$.
\textbf{Bài làm:}

\section{Bài 8}
Cho $|\phi\rangle=\frac{1}{4}|00\rangle+\frac{1}{2}|01\rangle+\frac{1}{\sqrt{2}}|10\rangle+\frac{\sqrt{3}}{4}|11\rangle$.

\subsection{(a)}
\textbf{Đề bài:} Tính $(H\otimes X)|\phi\rangle$.
\textbf{Bài làm:}

\subsection{(b)}
\textbf{Đề bài:} Tính $\mathrm{CNOT}|\phi\rangle$.
\textbf{Bài làm:}